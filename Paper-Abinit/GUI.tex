Human beings are usually involved in several tasks of Computational Physics 
from the writing of an input file to the final analysis of the results. As an 
important part of these tasks are done on remote high-performance 
computing centers, many of them are repetitive and time consuming.
AbinitGUI has been developed to help users to run the calculations in a more 
user-friendly environment and to automatize many repetitive tasks.

AbinitGUI allows people to work on a set of different machines, either locally 
or remotely. On top of the connection setup in the backend, the GUI provides a 
complete SSH terminal as well as a SFTP frame. Then, the typical steps of a 
calculation, i.e. pre-processing, actual running and analyzis of results, can 
be achieved with the help of graphical tools. In the pre-processing 
step, the input file is written inside the editor and then tested and 
analyzed thanks to an automatic parser generated from the YML documentation 
(\textcolor{red}{Presented in section ...}). The structure associated with 
each dataset can then be viewed in the embedded JMOL 
panel(\textcolor{red}{CITATION}).
As a second step, the user edits the configuration for the remote job 
management system and launches the job, all the files being automatically 
generated and submitted to the remote system.
Finally the user can tackle the analysis of the results through a 
set of post-processing scripts shared with the GUI, containing i.e. band 
structure plotters and links to the Python Library ``Abipy'' 
(\textcolor{red}{Presented in section ...}). 
A minimalistic interface is also provided to monitor the jobs (get 
current status, kill a specific job, ...).

\textcolor{red}{One should add a picture}

This project is developed in Java to be cross-platform and therefore does not 
require any installation. As the size of the package grew with time, the 
developers decided to extract the development from the main Abinit package and 
hosted the package on (\url{http://github.com/abinitgui/abinitgui}). A website 
has also been designed : \url{http://gui.abinit.org}, it contains the last 
stable version as well as documentation and supplementary information.