Running Abinit simulations, like for other ab-initio packages, requires to 
manually process lots of tasks ranging from the writing of an input file to 
the analysis of the results. As an 
important part of these tasks are launched on remote computing centers 
(clusters), many tasks are repetitive and time consuming.
AbinitGUI has been developed to help users to run the 
calculations in a more 
user-friendly environment and to automatize many repetitive tasks.

AbinitGUI allows one to launch simulations either locally 
or remotely. One of its handy features is to allow users to save several 
cluster configurations to send their simulation jobs. Furthermore, the GUI 
provides a complete SSH connection facility as well as a SSH terminal and a 
graphical SFTP client. A minimalistic interface is also provided to monitor the 
jobs (get current status, kill a specific job, ...).


During the pre-processing step, AbinitGUI allows one to write the input file 
inside 
it's built-in editor, but also to test and analyze it thanks to an automatic 
parser generated from the structured 
documentation written in YML format (\textcolor{red}{Presented in section 
...}). 
The geometry structure associated with each dataset can then be viewed in the 
embedded Jmol 
panel~\cite{Jmol}.
As a second step, the user edits the configuration for the specific remote job 
management system and launches the job. All the required files are
automatically 
generated and submitted to the remote system.

Finally the user can analyze the results through a 
set of post-processing scripts that are shipped with the GUI. The scripts are, 
for instance, band structure plotters, links to the Python Library ``Abipy'' 
(\textcolor{red}{Presented in section ...}), etc.

This project is developed in Java to be cross-platform and therefore does not 
require any installation. As the size of the package grew with time, the 
developers decided to extract the development from the main Abinit package and 
hosted the package on Github 
\cite{github-abinitgui}. A website 
has also been designed, it contains the last 
stable version as well as documentation and supplementary 
information~\cite{site-abinitgui}.